% Options for packages loaded elsewhere
\PassOptionsToPackage{unicode}{hyperref}
\PassOptionsToPackage{hyphens}{url}
%
\documentclass[
]{article}
\usepackage{amsmath,amssymb}
\usepackage{iftex}
\ifPDFTeX
  \usepackage[T1]{fontenc}
  \usepackage[utf8]{inputenc}
  \usepackage{textcomp} % provide euro and other symbols
\else % if luatex or xetex
  \usepackage{unicode-math} % this also loads fontspec
  \defaultfontfeatures{Scale=MatchLowercase}
  \defaultfontfeatures[\rmfamily]{Ligatures=TeX,Scale=1}
\fi
\usepackage{lmodern}
\ifPDFTeX\else
  % xetex/luatex font selection
\fi
% Use upquote if available, for straight quotes in verbatim environments
\IfFileExists{upquote.sty}{\usepackage{upquote}}{}
\IfFileExists{microtype.sty}{% use microtype if available
  \usepackage[]{microtype}
  \UseMicrotypeSet[protrusion]{basicmath} % disable protrusion for tt fonts
}{}
\makeatletter
\@ifundefined{KOMAClassName}{% if non-KOMA class
  \IfFileExists{parskip.sty}{%
    \usepackage{parskip}
  }{% else
    \setlength{\parindent}{0pt}
    \setlength{\parskip}{6pt plus 2pt minus 1pt}}
}{% if KOMA class
  \KOMAoptions{parskip=half}}
\makeatother
\usepackage{xcolor}
\usepackage[margin=1in]{geometry}
\usepackage{color}
\usepackage{fancyvrb}
\newcommand{\VerbBar}{|}
\newcommand{\VERB}{\Verb[commandchars=\\\{\}]}
\DefineVerbatimEnvironment{Highlighting}{Verbatim}{commandchars=\\\{\}}
% Add ',fontsize=\small' for more characters per line
\usepackage{framed}
\definecolor{shadecolor}{RGB}{248,248,248}
\newenvironment{Shaded}{\begin{snugshade}}{\end{snugshade}}
\newcommand{\AlertTok}[1]{\textcolor[rgb]{0.94,0.16,0.16}{#1}}
\newcommand{\AnnotationTok}[1]{\textcolor[rgb]{0.56,0.35,0.01}{\textbf{\textit{#1}}}}
\newcommand{\AttributeTok}[1]{\textcolor[rgb]{0.13,0.29,0.53}{#1}}
\newcommand{\BaseNTok}[1]{\textcolor[rgb]{0.00,0.00,0.81}{#1}}
\newcommand{\BuiltInTok}[1]{#1}
\newcommand{\CharTok}[1]{\textcolor[rgb]{0.31,0.60,0.02}{#1}}
\newcommand{\CommentTok}[1]{\textcolor[rgb]{0.56,0.35,0.01}{\textit{#1}}}
\newcommand{\CommentVarTok}[1]{\textcolor[rgb]{0.56,0.35,0.01}{\textbf{\textit{#1}}}}
\newcommand{\ConstantTok}[1]{\textcolor[rgb]{0.56,0.35,0.01}{#1}}
\newcommand{\ControlFlowTok}[1]{\textcolor[rgb]{0.13,0.29,0.53}{\textbf{#1}}}
\newcommand{\DataTypeTok}[1]{\textcolor[rgb]{0.13,0.29,0.53}{#1}}
\newcommand{\DecValTok}[1]{\textcolor[rgb]{0.00,0.00,0.81}{#1}}
\newcommand{\DocumentationTok}[1]{\textcolor[rgb]{0.56,0.35,0.01}{\textbf{\textit{#1}}}}
\newcommand{\ErrorTok}[1]{\textcolor[rgb]{0.64,0.00,0.00}{\textbf{#1}}}
\newcommand{\ExtensionTok}[1]{#1}
\newcommand{\FloatTok}[1]{\textcolor[rgb]{0.00,0.00,0.81}{#1}}
\newcommand{\FunctionTok}[1]{\textcolor[rgb]{0.13,0.29,0.53}{\textbf{#1}}}
\newcommand{\ImportTok}[1]{#1}
\newcommand{\InformationTok}[1]{\textcolor[rgb]{0.56,0.35,0.01}{\textbf{\textit{#1}}}}
\newcommand{\KeywordTok}[1]{\textcolor[rgb]{0.13,0.29,0.53}{\textbf{#1}}}
\newcommand{\NormalTok}[1]{#1}
\newcommand{\OperatorTok}[1]{\textcolor[rgb]{0.81,0.36,0.00}{\textbf{#1}}}
\newcommand{\OtherTok}[1]{\textcolor[rgb]{0.56,0.35,0.01}{#1}}
\newcommand{\PreprocessorTok}[1]{\textcolor[rgb]{0.56,0.35,0.01}{\textit{#1}}}
\newcommand{\RegionMarkerTok}[1]{#1}
\newcommand{\SpecialCharTok}[1]{\textcolor[rgb]{0.81,0.36,0.00}{\textbf{#1}}}
\newcommand{\SpecialStringTok}[1]{\textcolor[rgb]{0.31,0.60,0.02}{#1}}
\newcommand{\StringTok}[1]{\textcolor[rgb]{0.31,0.60,0.02}{#1}}
\newcommand{\VariableTok}[1]{\textcolor[rgb]{0.00,0.00,0.00}{#1}}
\newcommand{\VerbatimStringTok}[1]{\textcolor[rgb]{0.31,0.60,0.02}{#1}}
\newcommand{\WarningTok}[1]{\textcolor[rgb]{0.56,0.35,0.01}{\textbf{\textit{#1}}}}
\usepackage{longtable,booktabs,array}
\usepackage{calc} % for calculating minipage widths
% Correct order of tables after \paragraph or \subparagraph
\usepackage{etoolbox}
\makeatletter
\patchcmd\longtable{\par}{\if@noskipsec\mbox{}\fi\par}{}{}
\makeatother
% Allow footnotes in longtable head/foot
\IfFileExists{footnotehyper.sty}{\usepackage{footnotehyper}}{\usepackage{footnote}}
\makesavenoteenv{longtable}
\usepackage{graphicx}
\makeatletter
\def\maxwidth{\ifdim\Gin@nat@width>\linewidth\linewidth\else\Gin@nat@width\fi}
\def\maxheight{\ifdim\Gin@nat@height>\textheight\textheight\else\Gin@nat@height\fi}
\makeatother
% Scale images if necessary, so that they will not overflow the page
% margins by default, and it is still possible to overwrite the defaults
% using explicit options in \includegraphics[width, height, ...]{}
\setkeys{Gin}{width=\maxwidth,height=\maxheight,keepaspectratio}
% Set default figure placement to htbp
\makeatletter
\def\fps@figure{htbp}
\makeatother
\setlength{\emergencystretch}{3em} % prevent overfull lines
\providecommand{\tightlist}{%
  \setlength{\itemsep}{0pt}\setlength{\parskip}{0pt}}
\setcounter{secnumdepth}{-\maxdimen} % remove section numbering
\ifLuaTeX
  \usepackage{selnolig}  % disable illegal ligatures
\fi
\IfFileExists{bookmark.sty}{\usepackage{bookmark}}{\usepackage{hyperref}}
\IfFileExists{xurl.sty}{\usepackage{xurl}}{} % add URL line breaks if available
\urlstyle{same}
\hypersetup{
  pdftitle={Exploring Food Waste Data 🍏},
  pdfauthor={Josh Houlding},
  hidelinks,
  pdfcreator={LaTeX via pandoc}}

\title{Exploring Food Waste Data 🍏}
\author{Josh Houlding}
\date{2023-09-10}

\begin{document}
\maketitle

\hypertarget{introduction}{%
\section{Introduction 📚}\label{introduction}}

\href{https://en.wikipedia.org/wiki/Food_loss_and_waste}{Food waste}, a
pervasive global issue, refers to the disposal or wastage of edible food
that is still fit for consumption. This phenomenon not only squanders
precious resources but also has severe environmental, social, and
economic implications. It is estimated that nearly one-third of all food
produced for human consumption worldwide is lost or wasted. Such wastage
contributes to greenhouse gas emissions, depletes natural resources, and
exacerbates issues of food insecurity and hunger. Various factors
contribute to this problem, including consumer behavior, inadequate
storage, and inefficient supply chains. Understanding the complexities
surrounding food waste is crucial for implementing effective strategies
to mitigate its impact on both a local and global scale.

\hypertarget{objective}{%
\section{Objective 🎯}\label{objective}}

The primary objective of this project is to conduct a comprehensive
exploratory data analysis on a food waste dataset sourced from Kaggle.
Through this analysis, we aim to uncover valuable insights, identify
significant trends, and gain a deep understanding of the factors
contributing to food waste. Our goals include quantifying the extent of
food waste, examining its temporal and spatial distribution, identifying
key drivers of food waste, and exploring potential correlations with
various socioeconomic and environmental factors. Ultimately, this EDA
will enable us to make data-driven recommendations for reducing food
waste and promoting sustainable practices in the food industry.

\hypertarget{the-data}{%
\section{The data 📊}\label{the-data}}

\href{https://www.kaggle.com/datasets/joebeachcapital/food-waste}{This
dataset} contains information about food waste from every country
worldwide. It has been collected from different sources, such as
retailers, households, and restaurants. The data includes food waste
totals, both per-capita and absolute numbers, as well as totals for
household, retail and food service sources respectively. The dataset can
be used to analyze the patterns and causes of food waste, as well as to
propose solutions to reduce it. The dataset was created by Joe Beach
Capital, based on various public datasets available on Kaggle and other
platforms, and a
\href{https://informationisbeautiful.net/visualizations/food-waste/}{visualization
on informationisbeautiful.net} is also cited. This food waste data is
from 2021.

The dataset itself is a single CSV file containing food waste data from
214 countries.

\hypertarget{limitations}{%
\subsubsection{Limitations}\label{limitations}}

\begin{itemize}
\tightlist
\item
  The data is limited to 2021, so no temporal data is available for
  tracking trends over time.
\item
  The dataset does not include population data, so that will need to be
  acquired from another source at a later time if necessary.
\end{itemize}

\hypertarget{loading-the-data}{%
\section{Loading the data 🔄}\label{loading-the-data}}

\begin{Shaded}
\begin{Highlighting}[]
\CommentTok{\# Load packages}
\FunctionTok{library}\NormalTok{(tidyverse)}
\FunctionTok{library}\NormalTok{(dplyr)}
\FunctionTok{library}\NormalTok{(sqldf)}
\FunctionTok{library}\NormalTok{(ggplot2)}
\FunctionTok{library}\NormalTok{(readr)}
\FunctionTok{library}\NormalTok{(knitr)}

\CommentTok{\# Read the CSV file}
\NormalTok{foodWasteData }\OtherTok{\textless{}{-}} \FunctionTok{read\_csv}\NormalTok{(}\StringTok{"foodwastedata.csv"}\NormalTok{)}
\FunctionTok{kable}\NormalTok{(}\FunctionTok{head}\NormalTok{(foodWasteData), }\AttributeTok{caption=}\StringTok{"foodWasteData"}\NormalTok{)}
\end{Highlighting}
\end{Shaded}

\begin{longtable}[]{@{}
  >{\raggedright\arraybackslash}p{(\columnwidth - 22\tabcolsep) * \real{0.0504}}
  >{\raggedleft\arraybackslash}p{(\columnwidth - 22\tabcolsep) * \real{0.0856}}
  >{\raggedleft\arraybackslash}p{(\columnwidth - 22\tabcolsep) * \real{0.0907}}
  >{\raggedleft\arraybackslash}p{(\columnwidth - 22\tabcolsep) * \real{0.0831}}
  >{\raggedleft\arraybackslash}p{(\columnwidth - 22\tabcolsep) * \real{0.0831}}
  >{\raggedleft\arraybackslash}p{(\columnwidth - 22\tabcolsep) * \real{0.0756}}
  >{\raggedleft\arraybackslash}p{(\columnwidth - 22\tabcolsep) * \real{0.0982}}
  >{\raggedleft\arraybackslash}p{(\columnwidth - 22\tabcolsep) * \real{0.0907}}
  >{\raggedright\arraybackslash}p{(\columnwidth - 22\tabcolsep) * \real{0.0579}}
  >{\raggedleft\arraybackslash}p{(\columnwidth - 22\tabcolsep) * \real{0.0227}}
  >{\raggedright\arraybackslash}p{(\columnwidth - 22\tabcolsep) * \real{0.0806}}
  >{\raggedright\arraybackslash}p{(\columnwidth - 22\tabcolsep) * \real{0.1814}}@{}}
\caption{foodWasteData}\tabularnewline
\toprule\noalign{}
\begin{minipage}[b]{\linewidth}\raggedright
Country
\end{minipage} & \begin{minipage}[b]{\linewidth}\raggedleft
combined figures (kg/capita/year)
\end{minipage} & \begin{minipage}[b]{\linewidth}\raggedleft
Household estimate (kg/capita/year)
\end{minipage} & \begin{minipage}[b]{\linewidth}\raggedleft
Household estimate (tonnes/year)
\end{minipage} & \begin{minipage}[b]{\linewidth}\raggedleft
Retail estimate (kg/capita/year)
\end{minipage} & \begin{minipage}[b]{\linewidth}\raggedleft
Retail estimate (tonnes/year)
\end{minipage} & \begin{minipage}[b]{\linewidth}\raggedleft
Food service estimate (kg/capita/year)
\end{minipage} & \begin{minipage}[b]{\linewidth}\raggedleft
Food service estimate (tonnes/year)
\end{minipage} & \begin{minipage}[b]{\linewidth}\raggedright
Confidence in estimate
\end{minipage} & \begin{minipage}[b]{\linewidth}\raggedleft
M49 code
\end{minipage} & \begin{minipage}[b]{\linewidth}\raggedright
Region
\end{minipage} & \begin{minipage}[b]{\linewidth}\raggedright
Source
\end{minipage} \\
\midrule\noalign{}
\endfirsthead
\toprule\noalign{}
\begin{minipage}[b]{\linewidth}\raggedright
Country
\end{minipage} & \begin{minipage}[b]{\linewidth}\raggedleft
combined figures (kg/capita/year)
\end{minipage} & \begin{minipage}[b]{\linewidth}\raggedleft
Household estimate (kg/capita/year)
\end{minipage} & \begin{minipage}[b]{\linewidth}\raggedleft
Household estimate (tonnes/year)
\end{minipage} & \begin{minipage}[b]{\linewidth}\raggedleft
Retail estimate (kg/capita/year)
\end{minipage} & \begin{minipage}[b]{\linewidth}\raggedleft
Retail estimate (tonnes/year)
\end{minipage} & \begin{minipage}[b]{\linewidth}\raggedleft
Food service estimate (kg/capita/year)
\end{minipage} & \begin{minipage}[b]{\linewidth}\raggedleft
Food service estimate (tonnes/year)
\end{minipage} & \begin{minipage}[b]{\linewidth}\raggedright
Confidence in estimate
\end{minipage} & \begin{minipage}[b]{\linewidth}\raggedleft
M49 code
\end{minipage} & \begin{minipage}[b]{\linewidth}\raggedright
Region
\end{minipage} & \begin{minipage}[b]{\linewidth}\raggedright
Source
\end{minipage} \\
\midrule\noalign{}
\endhead
\bottomrule\noalign{}
\endlastfoot
Afghanistan & 126 & 82 & 3109153 & 16 & 594982 & 28 & 1051783 & Very Low
Confidence & 4 & Southern Asia &
\url{https://www.unep.org/resources/report/unep-food-waste-index-report-2151} \\
Albania & 127 & 83 & 238492 & 16 & 45058 & 28 & 79651 & Very Low
Confidence & 8 & Southern Europe &
\url{https://www.unep.org/resources/report/unep-food-waste-index-report-2160} \\
Algeria & 135 & 91 & 3918529 & 16 & 673360 & 28 & 1190335 & Very Low
Confidence & 12 & Northern Africa &
\url{https://www.unep.org/resources/report/unep-food-waste-index-report-2107} \\
Andorra & 123 & 84 & 6497 & 13 & 988 & 26 & 1971 & Low Confidence & 20 &
Southern Europe &
\url{https://www.unep.org/resources/report/unep-food-waste-index-report-2161} \\
Angola & 144 & 100 & 3169523 & 16 & 497755 & 28 & 879908 & Very Low
Confidence & 24 & Sub-Saharan Africa &
\url{https://www.unep.org/resources/report/unep-food-waste-index-report-2176} \\
Antigua and Barbuda & 113 & 74 & 7178 & 13 & 1244 & 26 & 2483 & Low
Confidence & 28 & Latin America and the Caribbean &
\url{https://www.unep.org/resources/report/unep-food-waste-index-report-2046} \\
\end{longtable}

\hypertarget{cleaning-the-data}{%
\section{Cleaning the data 🧹}\label{cleaning-the-data}}

\hypertarget{creating-a-new-column}{%
\subsubsection{Creating a new column}\label{creating-a-new-column}}

I created a new column for combined estimate in tonnes/year, since this
seemed important and there was no column for it in the original dataset.

\begin{Shaded}
\begin{Highlighting}[]
\CommentTok{\# Create new column for combined estimate in tonnes/year}
\NormalTok{foodWasteData }\OtherTok{\textless{}{-}} \FunctionTok{mutate}\NormalTok{(foodWasteData, }\AttributeTok{combined\_estimate\_tonnes\_per\_year =}\NormalTok{ foodWasteData}\SpecialCharTok{$}\StringTok{\textasciigrave{}}\AttributeTok{Household estimate (tonnes/year)}\StringTok{\textasciigrave{}} \SpecialCharTok{+}\NormalTok{ foodWasteData}\SpecialCharTok{$}\StringTok{\textasciigrave{}}\AttributeTok{Retail estimate (tonnes/year)}\StringTok{\textasciigrave{}} \SpecialCharTok{+}\NormalTok{ foodWasteData}\SpecialCharTok{$}\StringTok{\textasciigrave{}}\AttributeTok{Food service estimate (tonnes/year)}\StringTok{\textasciigrave{}}\NormalTok{)}
\NormalTok{foodWasteData }\OtherTok{\textless{}{-}}\NormalTok{ foodWasteData[, }\FunctionTok{c}\NormalTok{(}\DecValTok{1}\NormalTok{,}\DecValTok{2}\NormalTok{,}\DecValTok{13}\NormalTok{,}\DecValTok{3}\NormalTok{,}\DecValTok{4}\NormalTok{,}\DecValTok{5}\NormalTok{,}\DecValTok{6}\NormalTok{,}\DecValTok{7}\NormalTok{,}\DecValTok{8}\NormalTok{,}\DecValTok{9}\NormalTok{,}\DecValTok{10}\NormalTok{,}\DecValTok{11}\NormalTok{,}\DecValTok{12}\NormalTok{)]}
\end{Highlighting}
\end{Shaded}

\hypertarget{renaming-the-columns}{%
\subsubsection{Renaming the columns}\label{renaming-the-columns}}

I changed all column names to lowercase and put underscores in them so
they would be easier to feed into the sqldf package for the purposes of
running SQL queries on the data.

\begin{Shaded}
\begin{Highlighting}[]
\CommentTok{\# Changing column names to make them lowercase and include underscores}
\NormalTok{new\_col\_names }\OtherTok{\textless{}{-}} \FunctionTok{c}\NormalTok{(}\StringTok{"country"}\NormalTok{, }\StringTok{"combined\_estimate\_kpcpy"}\NormalTok{, }\StringTok{"combined\_estimate\_tpy"}\NormalTok{, }\StringTok{"household\_estimate\_kpcpy"}\NormalTok{, }\StringTok{"household\_estimate\_tpy"}\NormalTok{, }\StringTok{"retail\_estimate\_kpcpy"}\NormalTok{, }\StringTok{"retail\_estimate\_tpy"}\NormalTok{, }\StringTok{"food\_service\_estimate\_kpcpy"}\NormalTok{, }\StringTok{"food\_service\_estimate\_tpy"}\NormalTok{, }\StringTok{"estimate\_confidence"}\NormalTok{, }\StringTok{"m49\_code"}\NormalTok{, }\StringTok{"region"}\NormalTok{, }\StringTok{"source"}\NormalTok{)}
\FunctionTok{names}\NormalTok{(foodWasteData) }\OtherTok{\textless{}{-}}\NormalTok{ new\_col\_names}
\end{Highlighting}
\end{Shaded}

\hypertarget{moving-the-source-column-to-reduce-clutter}{%
\subsubsection{Moving the source column to reduce
clutter}\label{moving-the-source-column-to-reduce-clutter}}

I didn't need the source column for my analysis, so I decided to move it
instead of deleting it so it would be there if I needed it later on. The
links suggested the data is originally from the UN Environment
Programme, but the links in the dataset lead to pages where the data had
been removed for some reason.

\begin{Shaded}
\begin{Highlighting}[]
\CommentTok{\# Get source column out of the way}
\NormalTok{dataSources }\OtherTok{\textless{}{-}} \FunctionTok{sqldf}\NormalTok{(}\StringTok{"SELECT country, m49\_code, source FROM foodWasteData"}\NormalTok{)}
\NormalTok{foodWasteData}\OtherTok{\textless{}{-}}\NormalTok{ foodWasteData[, }\SpecialCharTok{{-}}\DecValTok{13}\NormalTok{]}
\end{Highlighting}
\end{Shaded}

\hypertarget{analyzing-the-data}{%
\section{Analyzing the data 🔬}\label{analyzing-the-data}}

\hypertarget{how-much-food-is-wasted-every-year-in-total}{%
\subsubsection{How much food is wasted every year in
total?}\label{how-much-food-is-wasted-every-year-in-total}}

\begin{Shaded}
\begin{Highlighting}[]
\CommentTok{\# Find total global food waste every year}
\NormalTok{total\_foodwaste }\OtherTok{\textless{}{-}} \FunctionTok{sqldf}\NormalTok{(}\StringTok{"SELECT SUM(combined\_estimate\_tpy) FROM foodWasteData"}\NormalTok{)[[}\DecValTok{1}\NormalTok{]]}
\FunctionTok{print}\NormalTok{(total\_foodwaste)}
\end{Highlighting}
\end{Shaded}

\begin{verbatim}
## [1] 930857271
\end{verbatim}

In total, a staggering 930,857,271 tonnes of food is wasted every year.
This is nearly a billion tonnes \emph{every single year}.

According to
\href{https://www.healthline.com/health/mens-health/average-weight-for-men}{Healthline},
the average adult weighs 136.7 pounds, or 62 kg.

Thus, our total annual food waste is equivalent to:

\begin{itemize}
\item ~
  \hypertarget{billion-human-adults.}{%
  \subparagraph{\texorpdfstring{\textbf{15,014,000,000 (\textgreater15
  billion) human
  adults.}}{15,014,000,000 (\textgreater15 billion) human adults.}}\label{billion-human-adults.}}
\item ~
  \hypertarget{million-2021-toyota-corollas.}{%
  \subparagraph{\texorpdfstring{\textbf{677,480,000 (\textgreater677
  million) 2021 Toyota
  Corollas.}}{677,480,000 (\textgreater677 million) 2021 Toyota Corollas.}}\label{million-2021-toyota-corollas.}}
\item ~
  \hypertarget{million-african-elephants.}{%
  \subparagraph{\texorpdfstring{\textbf{209,420,000 (\textgreater209
  million) African
  elephants.}}{209,420,000 (\textgreater209 million) African elephants.}}\label{million-african-elephants.}}
\item ~
  \hypertarget{american-nimitz-class-aircraft-carriers.}{%
  \subparagraph{\texorpdfstring{\textbf{9,402.6 American Nimitz-class
  aircraft
  carriers.}}{9,402.6 American Nimitz-class aircraft carriers.}}\label{american-nimitz-class-aircraft-carriers.}}
\end{itemize}

Truly staggering numbers in perspective.

\hypertarget{which-countries-have-the-highest-and-lowest-combined-per-capita-food-waste}{%
\subsubsection{Which countries have the highest and lowest combined
per-capita food
waste?}\label{which-countries-have-the-highest-and-lowest-combined-per-capita-food-waste}}

\begin{Shaded}
\begin{Highlighting}[]
\CommentTok{\# Find the country with the highest combined per{-}capita food waste}
\NormalTok{highestFoodWasteKpcpy }\OtherTok{\textless{}{-}} \FunctionTok{sqldf}\NormalTok{(}\StringTok{"SELECT country, MAX(combined\_estimate\_kpcpy), estimate\_confidence FROM foodWasteData"}\NormalTok{)}
\FunctionTok{kable}\NormalTok{(}\FunctionTok{head}\NormalTok{(highestFoodWasteKpcpy), }\AttributeTok{align=}\StringTok{"l"}\NormalTok{)}
\end{Highlighting}
\end{Shaded}

\begin{longtable}[]{@{}lll@{}}
\toprule\noalign{}
country & MAX(combined\_estimate\_kpcpy) & estimate\_confidence \\
\midrule\noalign{}
\endhead
\bottomrule\noalign{}
\endlastfoot
Malaysia & 260 & Medium Confidence \\
\end{longtable}

Malaysia has the highest combined per-capita food waste rate in the
world at 260 kg/capita/year.

\href{https://www.npr.org/sections/thesalt/2011/12/31/144478009/the-average-american-ate-literally-a-ton-this-year}{According
to NPR}, the average American eats 1,996 lbs (905.4 kg) of food per
year. Assuming that is 3 meals per day and people in every country eat
about the same amount annually, we end up with an estimate of 0.827
kg/meal.

Thus, the average Malaysian wastes a massive 314 meals' worth of food
every year.

\begin{Shaded}
\begin{Highlighting}[]
\CommentTok{\# Find the country with the lowest combined per{-}capita food waste}
\NormalTok{lowestFoodWasteKpcpy }\OtherTok{\textless{}{-}} \FunctionTok{sqldf}\NormalTok{(}\StringTok{"SELECT country, MIN(combined\_estimate\_kpcpy), estimate\_confidence FROM foodWasteData"}\NormalTok{)}
\FunctionTok{kable}\NormalTok{(}\FunctionTok{head}\NormalTok{(lowestFoodWasteKpcpy), }\AttributeTok{align=}\StringTok{"l"}\NormalTok{)}
\end{Highlighting}
\end{Shaded}

\begin{longtable}[]{@{}lll@{}}
\toprule\noalign{}
country & MIN(combined\_estimate\_kpcpy) & estimate\_confidence \\
\midrule\noalign{}
\endhead
\bottomrule\noalign{}
\endlastfoot
Slovenia & 61 & Medium Confidence \\
\end{longtable}

At the other end of the spectrum, Slovenia has the lowest combined
per-capita food waste of all countries, with its citizens wasting around
74 meals a year.

\hypertarget{what-are-the-top-10-and-bottom-10-countries-for-combined-per-capita-food-waste}{%
\subsubsection{What are the top 10 and bottom 10 countries for combined
per-capita food
waste?}\label{what-are-the-top-10-and-bottom-10-countries-for-combined-per-capita-food-waste}}

First, we aggregate the data.

\begin{Shaded}
\begin{Highlighting}[]
\CommentTok{\# Subset the dataframe to select the top 10 countries with the highest and lowest combined per{-}capita food waste}
\NormalTok{topCountries }\OtherTok{\textless{}{-}} \FunctionTok{tail}\NormalTok{(foodWasteData[}\FunctionTok{order}\NormalTok{(foodWasteData}\SpecialCharTok{$}\NormalTok{combined\_estimate\_kpcpy), ], }\DecValTok{10}\NormalTok{)}
\NormalTok{bottomCountries }\OtherTok{\textless{}{-}} \FunctionTok{head}\NormalTok{(foodWasteData[}\FunctionTok{order}\NormalTok{(foodWasteData}\SpecialCharTok{$}\NormalTok{combined\_estimate\_kpcpy), ], }\DecValTok{10}\NormalTok{)}
\end{Highlighting}
\end{Shaded}

Then display it:

\begin{Shaded}
\begin{Highlighting}[]
\CommentTok{\# Show the 10 countries with the highest combined per{-}capita food waste}
\FunctionTok{ggplot}\NormalTok{(topCountries, }\FunctionTok{aes}\NormalTok{(}\AttributeTok{x =} \FunctionTok{reorder}\NormalTok{(country, }\SpecialCharTok{{-}}\NormalTok{combined\_estimate\_kpcpy), }\AttributeTok{y =}\NormalTok{ combined\_estimate\_kpcpy)) }\SpecialCharTok{+}
  \FunctionTok{geom\_bar}\NormalTok{(}\AttributeTok{stat =} \StringTok{"identity"}\NormalTok{, }\AttributeTok{fill =} \StringTok{"red"}\NormalTok{, }\AttributeTok{width =} \FloatTok{0.7}\NormalTok{) }\SpecialCharTok{+}
  \FunctionTok{geom\_text}\NormalTok{(}\FunctionTok{aes}\NormalTok{(}\AttributeTok{label =}\NormalTok{ combined\_estimate\_kpcpy), }\AttributeTok{hjust =} \FloatTok{0.5}\NormalTok{, }\AttributeTok{vjust =} \SpecialCharTok{{-}}\FloatTok{0.3}\NormalTok{, }\AttributeTok{size =} \DecValTok{3}\NormalTok{) }\SpecialCharTok{+}  \CommentTok{\# Add labels}
  \FunctionTok{labs}\NormalTok{(}
    \AttributeTok{title =} \StringTok{"The 10 Countries with Highest Combined Per{-}Capita Food Waste"}\NormalTok{,}
    \AttributeTok{x =} \StringTok{"Country"}\NormalTok{,}
    \AttributeTok{y =} \StringTok{"Per{-}Capita Food Waste (kg/year)"}
\NormalTok{  ) }\SpecialCharTok{+}
  \FunctionTok{theme\_minimal}\NormalTok{() }\SpecialCharTok{+}
  \FunctionTok{theme}\NormalTok{(}\AttributeTok{axis.text.x =} \FunctionTok{element\_text}\NormalTok{(}\AttributeTok{angle =} \DecValTok{45}\NormalTok{, }\AttributeTok{hjust =} \DecValTok{1}\NormalTok{)) }\SpecialCharTok{+}
  \FunctionTok{scale\_fill\_manual}\NormalTok{(}
    \AttributeTok{values =} \FunctionTok{c}\NormalTok{(}\StringTok{"blue"} \OtherTok{=} \StringTok{"Combined"}\NormalTok{),}
    \AttributeTok{name =} \StringTok{"Waste Type"}\NormalTok{,}
    \AttributeTok{labels =} \FunctionTok{c}\NormalTok{(}\StringTok{"Combined"}\NormalTok{)}
\NormalTok{  ) }\SpecialCharTok{+}
  \FunctionTok{guides}\NormalTok{(}\AttributeTok{fill =} \ConstantTok{FALSE}\NormalTok{)  }\CommentTok{\# Hide the legend}
\end{Highlighting}
\end{Shaded}

\includegraphics{Exploring-Food-Waste-Data_files/figure-latex/unnamed-chunk-9-1.pdf}

We can see that the top 10 countries are largely located in Africa and
the Middle East, areas with lower levels of socioeconomic development.

\begin{Shaded}
\begin{Highlighting}[]
\CommentTok{\# Show the 10 countries with the lowest combined per{-}capita food waste}
\FunctionTok{ggplot}\NormalTok{(bottomCountries, }\FunctionTok{aes}\NormalTok{(}\AttributeTok{x =} \FunctionTok{reorder}\NormalTok{(country, }\SpecialCharTok{{-}}\NormalTok{combined\_estimate\_kpcpy), }\AttributeTok{y =}\NormalTok{ combined\_estimate\_kpcpy)) }\SpecialCharTok{+}
  \FunctionTok{geom\_bar}\NormalTok{(}\AttributeTok{stat =} \StringTok{"identity"}\NormalTok{, }\AttributeTok{fill =} \StringTok{"\#277a36"}\NormalTok{, }\AttributeTok{width =} \FloatTok{0.7}\NormalTok{) }\SpecialCharTok{+}
  \FunctionTok{geom\_text}\NormalTok{(}\FunctionTok{aes}\NormalTok{(}\AttributeTok{label =}\NormalTok{ combined\_estimate\_kpcpy), }\AttributeTok{hjust =} \FloatTok{0.5}\NormalTok{, }\AttributeTok{vjust =} \SpecialCharTok{{-}}\FloatTok{0.3}\NormalTok{, }\AttributeTok{size =} \DecValTok{3}\NormalTok{) }\SpecialCharTok{+}  \CommentTok{\# Add labels}
  \FunctionTok{labs}\NormalTok{(}
    \AttributeTok{title =} \StringTok{"The 10 Countries with Lowest Combined Per{-}Capita Food Waste"}\NormalTok{,}
    \AttributeTok{x =} \StringTok{"Country"}\NormalTok{,}
    \AttributeTok{y =} \StringTok{"Per{-}Capita Food Waste (kg/year)"}
\NormalTok{  ) }\SpecialCharTok{+}
  \FunctionTok{theme\_minimal}\NormalTok{() }\SpecialCharTok{+}
  \FunctionTok{theme}\NormalTok{(}\AttributeTok{axis.text.x =} \FunctionTok{element\_text}\NormalTok{(}\AttributeTok{angle =} \DecValTok{45}\NormalTok{, }\AttributeTok{hjust =} \DecValTok{1}\NormalTok{)) }\SpecialCharTok{+}
  \FunctionTok{scale\_fill\_manual}\NormalTok{(}
    \AttributeTok{values =} \FunctionTok{c}\NormalTok{(}\StringTok{"blue"} \OtherTok{=} \StringTok{"Combined"}\NormalTok{),}
    \AttributeTok{name =} \StringTok{"Waste Type"}\NormalTok{,}
    \AttributeTok{labels =} \FunctionTok{c}\NormalTok{(}\StringTok{"Combined"}\NormalTok{)}
\NormalTok{  ) }\SpecialCharTok{+}
  \FunctionTok{guides}\NormalTok{(}\AttributeTok{fill =} \ConstantTok{FALSE}\NormalTok{)  }\CommentTok{\# Hide the legend}
\end{Highlighting}
\end{Shaded}

\includegraphics{Exploring-Food-Waste-Data_files/figure-latex/unnamed-chunk-10-1.pdf}

As expected, most of the countries with the lowest food waste per capita
are in Europe, a highly-developed region of the world.

\hypertarget{which-countries-generate-the-most-and-least-combined-total-food-waste}{%
\subsubsection{Which countries generate the most and least combined
total food
waste?}\label{which-countries-generate-the-most-and-least-combined-total-food-waste}}

\begin{Shaded}
\begin{Highlighting}[]
\CommentTok{\# Find the country with the highest combined total food waste}
\NormalTok{highestFoodWaste }\OtherTok{\textless{}{-}} \FunctionTok{sqldf}\NormalTok{(}\StringTok{"SELECT country, MAX(combined\_estimate\_tpy), estimate\_confidence FROM foodWasteData"}\NormalTok{)}
\FunctionTok{kable}\NormalTok{(}\FunctionTok{head}\NormalTok{(highestFoodWaste), }\AttributeTok{align=}\StringTok{"l"}\NormalTok{)}
\end{Highlighting}
\end{Shaded}

\begin{longtable}[]{@{}lll@{}}
\toprule\noalign{}
country & MAX(combined\_estimate\_tpy) & estimate\_confidence \\
\midrule\noalign{}
\endhead
\bottomrule\noalign{}
\endlastfoot
China & 179448659 & Very Low Confidence \\
\end{longtable}

The country with the highest combined total food waste estimate is
China, at over 179 million tonnes. However, this estimate could be
inaccurate as very low confidence is given.

\begin{Shaded}
\begin{Highlighting}[]
\CommentTok{\# Find the country with the lowest combined total food waste}
\NormalTok{lowestFoodWaste }\OtherTok{\textless{}{-}} \FunctionTok{sqldf}\NormalTok{(}\StringTok{"SELECT country, MIN(combined\_estimate\_tpy), estimate\_confidence FROM foodWasteData"}\NormalTok{)}
\FunctionTok{kable}\NormalTok{(}\FunctionTok{head}\NormalTok{(lowestFoodWaste), }\AttributeTok{align=}\StringTok{"l"}\NormalTok{)}
\end{Highlighting}
\end{Shaded}

\begin{longtable}[]{@{}lll@{}}
\toprule\noalign{}
country & MIN(combined\_estimate\_tpy) & estimate\_confidence \\
\midrule\noalign{}
\endhead
\bottomrule\noalign{}
\endlastfoot
Nauru & 1264 & Low Confidence \\
\end{longtable}

Nauru is a small Micronesian island with a tiny population of 12,511 as
of 2021, so it is expected that they would have a miniscule total amount
of food waste.

\hypertarget{what-is-the-average-combined-per-capita-food-waste-across-all-countries}{%
\subsubsection{What is the average combined per-capita food waste across
all
countries?}\label{what-is-the-average-combined-per-capita-food-waste-across-all-countries}}

\begin{Shaded}
\begin{Highlighting}[]
\CommentTok{\# Find the average combined per{-}capita food waste across all countries}
\NormalTok{avg\_foodwaste\_kpcpy }\OtherTok{\textless{}{-}} \FunctionTok{sqldf}\NormalTok{(}\StringTok{"SELECT AVG(combined\_estimate\_kpcpy) FROM foodWasteData"}\NormalTok{)[[}\DecValTok{1}\NormalTok{]]}
\FunctionTok{print}\NormalTok{(avg\_foodwaste\_kpcpy)}
\end{Highlighting}
\end{Shaded}

\begin{verbatim}
## [1] 126.7944
\end{verbatim}

The average person on Earth wastes about 126.8 kg of food every year,
which is about 153 meals' worth given our estimate of 0.827 kg/meal
earlier.

\hypertarget{how-does-average-per-capita-food-waste-vary-across-regions}{%
\subsubsection{How does average per-capita food waste vary across
regions?}\label{how-does-average-per-capita-food-waste-vary-across-regions}}

I wanted to explore the variation in per-capita food waste across
regions, and not just individual countries.

\begin{Shaded}
\begin{Highlighting}[]
\CommentTok{\# Display all regions in the dataset}
\FunctionTok{print}\NormalTok{(}\FunctionTok{unique}\NormalTok{(foodWasteData}\SpecialCharTok{$}\NormalTok{region))}
\end{Highlighting}
\end{Shaded}

\begin{verbatim}
##  [1] "Southern Asia"                   "Southern Europe"                
##  [3] "Northern Africa"                 "Sub-Saharan Africa"             
##  [5] "Latin America and the Caribbean" "Western Asia"                   
##  [7] "Australia and New Zealand"       "Western Europe"                 
##  [9] "Eastern Europe"                  "Northern America"               
## [11] "South-eastern Asia"              "Eastern Asia"                   
## [13] "Northern Europe"                 "Melanesia"                      
## [15] "Polynesia"                       "Micronesia"                     
## [17] "Central Asia"
\end{verbatim}

\begin{Shaded}
\begin{Highlighting}[]
\CommentTok{\# Group the data by the \textquotesingle{}region\textquotesingle{} column}
\NormalTok{foodWasteData }\OtherTok{\textless{}{-}}\NormalTok{ foodWasteData }\SpecialCharTok{\%\textgreater{}\%}
  \FunctionTok{group\_by}\NormalTok{(region)}
\CommentTok{\# Calculate averages for each per{-}capita food waste column}
\NormalTok{averageFoodWasteByRegion }\OtherTok{\textless{}{-}}\NormalTok{ foodWasteData }\SpecialCharTok{\%\textgreater{}\%}
  \FunctionTok{summarize}\NormalTok{(}
    \AttributeTok{avg\_combined\_kpcpy =} \FunctionTok{mean}\NormalTok{(combined\_estimate\_kpcpy, }\AttributeTok{na.rm =} \ConstantTok{TRUE}\NormalTok{),}
    \AttributeTok{avg\_household\_kpcpy =} \FunctionTok{mean}\NormalTok{(household\_estimate\_kpcpy, }\AttributeTok{na.rm =} \ConstantTok{TRUE}\NormalTok{),}
    \AttributeTok{avg\_retail\_kpcpy =} \FunctionTok{mean}\NormalTok{(retail\_estimate\_kpcpy, }\AttributeTok{na.rm =} \ConstantTok{TRUE}\NormalTok{),}
    \AttributeTok{avg\_food\_service\_kpcpy =} \FunctionTok{mean}\NormalTok{(food\_service\_estimate\_kpcpy, }\AttributeTok{na.rm =} \ConstantTok{TRUE}\NormalTok{)}
\NormalTok{    )}
\CommentTok{\# Order the data by combined averages}
\NormalTok{averageFoodWasteByRegion }\OtherTok{\textless{}{-}}\NormalTok{ averageFoodWasteByRegion[}\FunctionTok{order}\NormalTok{(averageFoodWasteByRegion}\SpecialCharTok{$}\NormalTok{avg\_combined\_kpcpy, }\AttributeTok{decreasing =} \ConstantTok{TRUE}\NormalTok{), ]}
\FunctionTok{kable}\NormalTok{(}\FunctionTok{head}\NormalTok{(averageFoodWasteByRegion, }\DecValTok{17}\NormalTok{), }\AttributeTok{align=}\StringTok{"l"}\NormalTok{)}
\end{Highlighting}
\end{Shaded}

\begin{longtable}[]{@{}
  >{\raggedright\arraybackslash}p{(\columnwidth - 8\tabcolsep) * \real{0.2883}}
  >{\raggedright\arraybackslash}p{(\columnwidth - 8\tabcolsep) * \real{0.1712}}
  >{\raggedright\arraybackslash}p{(\columnwidth - 8\tabcolsep) * \real{0.1802}}
  >{\raggedright\arraybackslash}p{(\columnwidth - 8\tabcolsep) * \real{0.1532}}
  >{\raggedright\arraybackslash}p{(\columnwidth - 8\tabcolsep) * \real{0.2072}}@{}}
\toprule\noalign{}
\begin{minipage}[b]{\linewidth}\raggedright
region
\end{minipage} & \begin{minipage}[b]{\linewidth}\raggedright
avg\_combined\_kpcpy
\end{minipage} & \begin{minipage}[b]{\linewidth}\raggedright
avg\_household\_kpcpy
\end{minipage} & \begin{minipage}[b]{\linewidth}\raggedright
avg\_retail\_kpcpy
\end{minipage} & \begin{minipage}[b]{\linewidth}\raggedright
avg\_food\_service\_kpcpy
\end{minipage} \\
\midrule\noalign{}
\endhead
\bottomrule\noalign{}
\endlastfoot
Sub-Saharan Africa & 145.3958 & 101.64583 & 15.77083 & 27.97917 \\
Western Asia & 145.3889 & 101.05556 & 17.16667 & 27.16667 \\
South-eastern Asia & 137.4545 & 83.00000 & 21.18182 & 33.27273 \\
Northern Africa & 133.5000 & 89.50000 & 16.00000 & 28.00000 \\
Central Asia & 130.2000 & 86.20000 & 16.00000 & 28.00000 \\
Melanesia & 128.6000 & 85.60000 & 15.40000 & 27.60000 \\
Micronesia & 123.1429 & 82.00000 & 14.28571 & 26.85714 \\
Southern Europe & 123.0000 & 85.60000 & 12.60000 & 24.80000 \\
Northern America & 120.7500 & 71.50000 & 13.75000 & 35.50000 \\
Polynesia & 119.5000 & 76.75000 & 15.25000 & 27.50000 \\
Latin America and the Caribbean & 115.5476 & 73.57143 & 14.78571 &
27.19048 \\
Southern Asia & 113.1111 & 71.88889 & 16.00000 & 25.22222 \\
Eastern Asia & 113.0000 & 71.42857 & 13.71429 & 27.85714 \\
Northern Europe & 111.6667 & 74.66667 & 12.83333 & 24.16667 \\
Australia and New Zealand & 111.5000 & 81.50000 & 6.00000 & 24.00000 \\
Eastern Europe & 109.4000 & 68.10000 & 14.30000 & 27.00000 \\
Western Europe & 104.8889 & 67.11111 & 12.00000 & 25.77778 \\
\end{longtable}

The three regions with the highest per-capita food waste are Sub-Saharan
Africa, Western Asia and South-eastern Asia, at 145.4 kg/person/year for
the former two and 137.5 kg/person/year for the latter. On the other
hand, Western Europe is a very wealthy region and has the lowest
per-capita food waste, at 104.9 kg/person/year. Northern America is near
the middle of the pack, at 9th out of 17 regions. This was surprising,
as I predicted that more economically-developed regions of the world
would waste more. I was curious about this and wanted to learn more.

According to the
\href{https://www.wri.org/insights/reducing-food-loss-and-food-waste\#:~:text=There\%20used\%20to\%20be\%20a,shown\%20this\%20isn\%27t\%20true.}{World
Resources Institute}, poor infrastructure (unreliable roads, lack of
cold storage, old equipment), suboptimal packaging, poor food
management, and consumer behaviors are the most impactful factors on
food waste. Less-developed regions like Sub-Saharan Africa may lack
dependable roads and high-tech farming equipment that enable food to get
from farm to table before going bad, as well as proper cold storage and
packaging to keep food edible. Consumers in these regions may also be
unaware of efficient food preparation and storage techniques, leading to
tonnes of unnecessary waste.

Additional fun fact: ``The environmental impact of wasted food is
greater than that of packaging waste. So, while it's important to limit
this waste, it's also important to use correct packaging to reduce food
spoilage'' (Goodwin, 2023). In other words, using more packaging to make
sure food doesn't go off is a worthwhile investment from an
environmental perspective.

\hypertarget{how-does-household-per-capita-food-waste-compare-to-retail-and-food-service-per-capita-food-waste}{%
\subsubsection{How does household per-capita food waste compare to
retail and food service per-capita food
waste?}\label{how-does-household-per-capita-food-waste-compare-to-retail-and-food-service-per-capita-food-waste}}

Let's take a look at the data from above in a graph.

\begin{Shaded}
\begin{Highlighting}[]
\CommentTok{\# Show how different categories of per{-}capita food waste compare in each region}
\FunctionTok{ggplot}\NormalTok{(averageFoodWasteByRegion, }\FunctionTok{aes}\NormalTok{(}\AttributeTok{x =} \FunctionTok{reorder}\NormalTok{(region, avg\_combined\_kpcpy), }\AttributeTok{y =}\NormalTok{ avg\_combined\_kpcpy, }\AttributeTok{fill =} \StringTok{"Combined"}\NormalTok{)) }\SpecialCharTok{+}
  \FunctionTok{geom\_bar}\NormalTok{(}\AttributeTok{stat =} \StringTok{"identity"}\NormalTok{, }\AttributeTok{fill =} \StringTok{"blue"}\NormalTok{, }\AttributeTok{width =} \FloatTok{0.7}\NormalTok{) }\SpecialCharTok{+}
  \FunctionTok{geom\_col}\NormalTok{(}\FunctionTok{aes}\NormalTok{(}\AttributeTok{y =}\NormalTok{ avg\_household\_kpcpy, }\AttributeTok{fill =} \StringTok{"Household"}\NormalTok{), }\AttributeTok{width =} \FloatTok{0.7}\NormalTok{) }\SpecialCharTok{+}
  \FunctionTok{geom\_col}\NormalTok{(}\FunctionTok{aes}\NormalTok{(}\AttributeTok{y =}\NormalTok{ avg\_retail\_kpcpy, }\AttributeTok{fill =} \StringTok{"Retail"}\NormalTok{), }\AttributeTok{width =} \FloatTok{0.7}\NormalTok{) }\SpecialCharTok{+}
  \FunctionTok{geom\_col}\NormalTok{(}\FunctionTok{aes}\NormalTok{(}\AttributeTok{y =}\NormalTok{ avg\_food\_service\_kpcpy, }\AttributeTok{fill =} \StringTok{"Food Service"}\NormalTok{), }\AttributeTok{width =} \FloatTok{0.7}\NormalTok{) }\SpecialCharTok{+}
  \FunctionTok{labs}\NormalTok{(}
    \AttributeTok{title =} \StringTok{"Relative Proportions of Food Waste Types by Region"}\NormalTok{,}
    \AttributeTok{x =} \StringTok{"Region"}\NormalTok{,}
    \AttributeTok{y =} \StringTok{"Average Per{-}Capita Food Waste (kg/year)"}
\NormalTok{  ) }\SpecialCharTok{+}
  \FunctionTok{theme\_minimal}\NormalTok{() }\SpecialCharTok{+}
  \FunctionTok{theme}\NormalTok{(}\AttributeTok{axis.text.x =} \FunctionTok{element\_text}\NormalTok{(}\AttributeTok{angle =} \DecValTok{0}\NormalTok{, }\AttributeTok{hjust =} \FloatTok{0.5}\NormalTok{)) }\SpecialCharTok{+}
  \FunctionTok{scale\_fill\_manual}\NormalTok{(}
    \AttributeTok{values =} \FunctionTok{c}\NormalTok{(}\StringTok{"Household"} \OtherTok{=} \StringTok{"green"}\NormalTok{, }\StringTok{"Retail"} \OtherTok{=} \StringTok{"blue"}\NormalTok{, }\StringTok{"Food Service"} \OtherTok{=} \StringTok{"orange"}\NormalTok{),}
    \AttributeTok{name =} \StringTok{"Food Waste Types"}\NormalTok{,}
    \AttributeTok{labels =} \FunctionTok{c}\NormalTok{(}\StringTok{"Retail"}\NormalTok{, }\StringTok{"Household"}\NormalTok{, }\StringTok{"Food Service"}\NormalTok{)}
\NormalTok{  ) }\SpecialCharTok{+}
  \FunctionTok{guides}\NormalTok{(}\AttributeTok{fill =} \FunctionTok{guide\_legend}\NormalTok{(}\AttributeTok{title =} \StringTok{"Food Waste Types"}\NormalTok{)) }\SpecialCharTok{+}
  \FunctionTok{coord\_flip}\NormalTok{()  }\CommentTok{\# Flip the x and y axes for horizontal bars}
\end{Highlighting}
\end{Shaded}

\includegraphics{Exploring-Food-Waste-Data_files/figure-latex/unnamed-chunk-15-1.pdf}

Clearly, the biggest contributor to food waste is households for every
region, and the disparity between household food waste and the other two
sources is especially striking in less-developed regions like
Sub-Saharan Africa and Western Asia. More developed regions like Eastern
and Western Europe have a more uniform distribution of food waste
contributions from all 3 sources.

\hypertarget{which-source-contributes-the-most-to-total-food-waste}{%
\subsubsection{Which source contributes the most to total food
waste?}\label{which-source-contributes-the-most-to-total-food-waste}}

Knowing the proportions of food waste from household, retail and food
service in every region, I wanted to see how the totals compared.

\begin{Shaded}
\begin{Highlighting}[]
\CommentTok{\# Compare food waste output of households, retail and food service}
\NormalTok{totalFoodWaste }\OtherTok{\textless{}{-}} \FunctionTok{sqldf}\NormalTok{(}\StringTok{"SELECT SUM(household\_estimate\_tpy) AS total\_household, SUM(retail\_estimate\_tpy) AS total\_retail, SUM(food\_service\_estimate\_tpy) AS total\_food\_service FROM foodWasteData"}\NormalTok{)}
\FunctionTok{kable}\NormalTok{(}\FunctionTok{head}\NormalTok{(totalFoodWaste), }\AttributeTok{align=}\StringTok{"l"}\NormalTok{)}
\end{Highlighting}
\end{Shaded}

\begin{longtable}[]{@{}lll@{}}
\toprule\noalign{}
total\_household & total\_retail & total\_food\_service \\
\midrule\noalign{}
\endhead
\bottomrule\noalign{}
\endlastfoot
569003670 & 118137722 & 243715879 \\
\end{longtable}

\begin{Shaded}
\begin{Highlighting}[]
\CommentTok{\# Visualize total food waste broken down by source}
\FunctionTok{ggplot}\NormalTok{() }\SpecialCharTok{+}
  \FunctionTok{geom\_bar}\NormalTok{(}
    \FunctionTok{aes}\NormalTok{(}\AttributeTok{x =} \FunctionTok{c}\NormalTok{(}\StringTok{"Household"}\NormalTok{, }\StringTok{"Retail"}\NormalTok{, }\StringTok{"Food Service"}\NormalTok{), }
        \AttributeTok{y =} \FunctionTok{c}\NormalTok{(}\FunctionTok{as.numeric}\NormalTok{(totalFoodWaste[}\DecValTok{1}\NormalTok{,}\DecValTok{1}\NormalTok{]), }
              \FunctionTok{as.numeric}\NormalTok{(totalFoodWaste[}\DecValTok{1}\NormalTok{,}\DecValTok{2}\NormalTok{]), }
              \FunctionTok{as.numeric}\NormalTok{(totalFoodWaste[}\DecValTok{1}\NormalTok{,}\DecValTok{3}\NormalTok{])), }
        \AttributeTok{fill =} \FunctionTok{c}\NormalTok{(}\StringTok{"green"}\NormalTok{, }\StringTok{"orange"}\NormalTok{, }\StringTok{"blue"}\NormalTok{)),}
    \AttributeTok{stat =} \StringTok{"identity"}\NormalTok{, }
    \AttributeTok{position =} \StringTok{"dodge"}\NormalTok{, }
    \AttributeTok{width =} \FloatTok{0.7}
\NormalTok{  ) }\SpecialCharTok{+}
  \FunctionTok{geom\_text}\NormalTok{(}
    \FunctionTok{aes}\NormalTok{(}\AttributeTok{x =} \FunctionTok{c}\NormalTok{(}\StringTok{"Household"}\NormalTok{, }\StringTok{"Retail"}\NormalTok{, }\StringTok{"Food Service"}\NormalTok{), }
        \AttributeTok{y =} \FunctionTok{c}\NormalTok{(}\FunctionTok{as.numeric}\NormalTok{(totalFoodWaste[}\DecValTok{1}\NormalTok{,}\DecValTok{1}\NormalTok{]), }
              \FunctionTok{as.numeric}\NormalTok{(totalFoodWaste[}\DecValTok{1}\NormalTok{,}\DecValTok{2}\NormalTok{]), }
              \FunctionTok{as.numeric}\NormalTok{(totalFoodWaste[}\DecValTok{1}\NormalTok{,}\DecValTok{3}\NormalTok{])), }
        \AttributeTok{label =}\NormalTok{ scales}\SpecialCharTok{::}\FunctionTok{comma}\NormalTok{(}\FunctionTok{c}\NormalTok{(}\FunctionTok{as.numeric}\NormalTok{(totalFoodWaste[}\DecValTok{1}\NormalTok{,}\DecValTok{1}\NormalTok{]), }
                                \FunctionTok{as.numeric}\NormalTok{(totalFoodWaste[}\DecValTok{1}\NormalTok{,}\DecValTok{2}\NormalTok{]), }
                                \FunctionTok{as.numeric}\NormalTok{(totalFoodWaste[}\DecValTok{1}\NormalTok{,}\DecValTok{3}\NormalTok{])))),}
    \AttributeTok{position =} \FunctionTok{position\_dodge}\NormalTok{(}\AttributeTok{width =} \FloatTok{0.7}\NormalTok{),}
    \AttributeTok{vjust =} \SpecialCharTok{{-}}\FloatTok{0.5}\NormalTok{, }\AttributeTok{size =} \DecValTok{3}
\NormalTok{  ) }\SpecialCharTok{+}
  \FunctionTok{labs}\NormalTok{(}
    \AttributeTok{title =} \StringTok{"Total Food Waste by Source"}\NormalTok{,}
    \AttributeTok{x =} \StringTok{"Source"}\NormalTok{,}
    \AttributeTok{y =} \StringTok{"Total Food Waste (kg)"}
\NormalTok{  ) }\SpecialCharTok{+}
  \FunctionTok{scale\_fill\_manual}\NormalTok{(}
    \AttributeTok{values =} \FunctionTok{c}\NormalTok{(}\StringTok{"orange"} \OtherTok{=} \StringTok{"orange"}\NormalTok{, }\StringTok{"green"} \OtherTok{=} \StringTok{"green"}\NormalTok{, }\StringTok{"blue"} \OtherTok{=} \StringTok{"blue"}\NormalTok{),}
    \AttributeTok{name =} \StringTok{"Food Waste Types"}\NormalTok{,}
    \AttributeTok{breaks =} \FunctionTok{c}\NormalTok{(}\StringTok{"orange"}\NormalTok{, }\StringTok{"green"}\NormalTok{, }\StringTok{"blue"}\NormalTok{),}
    \AttributeTok{labels =} \FunctionTok{c}\NormalTok{(}\StringTok{"Retail"}\NormalTok{, }\StringTok{"Household"}\NormalTok{, }\StringTok{"Food Service"}\NormalTok{)}
\NormalTok{  ) }\SpecialCharTok{+}
  \FunctionTok{theme\_minimal}\NormalTok{()}
\end{Highlighting}
\end{Shaded}

\includegraphics{Exploring-Food-Waste-Data_files/figure-latex/unnamed-chunk-17-1.pdf}

Household food waste far surpasses that of food service and retail put
together. This could be happening for a variety of reasons. Consumers
could be accidentally purchasing too much food because of bulk buying,
lack of meal planning or impulse buying, and throwing a lot of it out.
They could also be unaware of the severe environmental and economic
consequences of food waste, leading to wasteful behavior. As previously
mentioned, many people in developing regions may not have access to
refrigerators or other effective food storage units, leading to food
going bad before it can be used. In some cultures, having an abundance
of food is seen as a sign of hospitality or generosity, which can lead
to more waste.

On the retail and food service side, many retailers around the world may
reject imperfect produce or packaged food and throw it out instead of
donating it. Food service establishments may have similar practices,
such as pizza shops throwing out display slices at the end of the day
instead of keeping or donating them.

\hypertarget{is-there-a-relationship-between-a-countrys-population-size-and-its-total-food-waste}{%
\subsubsection{Is there a relationship between a country's population
size and its total food
waste?}\label{is-there-a-relationship-between-a-countrys-population-size-and-its-total-food-waste}}

Answering this question required finding a dataset with population
numbers for each country. However, joining it directly to the food waste
dataset would be difficult because every dataset I found only had
ISO-alpha3 3-letter country codes, and \texttt{foodWasteData}only had
M49 country codes. I then had the idea of joining \texttt{foodWasteData}
with a country code dataset containing both M49 codes \emph{and}
ISO-alpha3 codes by M49 codes, then joining the updated
\texttt{foodWasteData} with a new population dataset by ISO-alpha3
codes. I found the following two datasets on Kaggle to accomplish this
task:

\begin{itemize}
\tightlist
\item
  \href{https://www.kaggle.com/datasets/najielkotob/standard-country-or-area-codes-m49}{Standard
  country or area codes (M49) \textbar{} Kaggle}
\item
  \href{https://www.kaggle.com/datasets/rsrishav/world-population}{2021
  World Population (updated daily) \textbar{} Kaggle}
\end{itemize}

First, we join the datasets:

\begin{Shaded}
\begin{Highlighting}[]
\CommentTok{\# Load dataset linking M49 codes to ISO{-}alpha3 codes and join with foodWasteData}
\NormalTok{codesData }\OtherTok{\textless{}{-}} \FunctionTok{read\_csv}\NormalTok{(}\StringTok{"M49 and ISO{-}alpha3 codes.csv"}\NormalTok{)}
\FunctionTok{names}\NormalTok{(codesData)[}\FunctionTok{names}\NormalTok{(codesData) }\SpecialCharTok{==} \StringTok{\textquotesingle{}M49 code\textquotesingle{}}\NormalTok{] }\OtherTok{\textless{}{-}} \StringTok{\textquotesingle{}m49\_code\textquotesingle{}}
\FunctionTok{names}\NormalTok{(codesData)[}\FunctionTok{names}\NormalTok{(codesData) }\SpecialCharTok{==} \StringTok{\textquotesingle{}ISO{-}alpha3 code\textquotesingle{}}\NormalTok{] }\OtherTok{\textless{}{-}} \StringTok{\textquotesingle{}iso\_alpha3\_code\textquotesingle{}}
\NormalTok{codesData }\OtherTok{\textless{}{-}} \FunctionTok{sqldf}\NormalTok{(}\StringTok{"SELECT m49\_code, iso\_alpha3\_code FROM codesData"}\NormalTok{)}
\NormalTok{foodWasteData }\OtherTok{\textless{}{-}} \FunctionTok{sqldf}\NormalTok{(}\StringTok{"SELECT * FROM foodWasteData INNER JOIN codesData }
\StringTok{                 ON foodWasteData.m49\_code = codesData.m49\_code"}\NormalTok{)}

\CommentTok{\# Join population dataset with foodWasteData using ISO{-}alpha3 codes}
\NormalTok{populationData }\OtherTok{\textless{}{-}} \FunctionTok{read\_csv}\NormalTok{(}\StringTok{"2021\_population.csv"}\NormalTok{)}
\FunctionTok{names}\NormalTok{(populationData)[}\FunctionTok{names}\NormalTok{(populationData) }\SpecialCharTok{==} \StringTok{\textquotesingle{}iso\_code\textquotesingle{}}\NormalTok{] }\OtherTok{\textless{}{-}} \StringTok{\textquotesingle{}iso\_alpha3\_code\textquotesingle{}}
\FunctionTok{names}\NormalTok{(populationData)[}\FunctionTok{names}\NormalTok{(populationData) }\SpecialCharTok{==} \StringTok{\textquotesingle{}2021\_last\_updated\textquotesingle{}}\NormalTok{] }\OtherTok{\textless{}{-}} \StringTok{\textquotesingle{}population\_2021\textquotesingle{}}
\NormalTok{populationData }\OtherTok{\textless{}{-}} \FunctionTok{sqldf}\NormalTok{(}\StringTok{"SELECT iso\_alpha3\_code, population\_2021 FROM populationData"}\NormalTok{)}
\NormalTok{foodWasteData }\OtherTok{\textless{}{-}} \FunctionTok{merge}\NormalTok{(foodWasteData, populationData, }\AttributeTok{by =} \StringTok{"iso\_alpha3\_code"}\NormalTok{, }\AttributeTok{all.x =} \ConstantTok{TRUE}\NormalTok{)}
\CommentTok{\# Reorder foodWasteData and remove unnecessary duplicate m49 column}
\NormalTok{foodWasteData }\OtherTok{\textless{}{-}}\NormalTok{ foodWasteData[, }\SpecialCharTok{{-}}\DecValTok{14}\NormalTok{]}
\NormalTok{foodWasteData }\OtherTok{\textless{}{-}}\NormalTok{ foodWasteData[, }\FunctionTok{c}\NormalTok{(}\DecValTok{2}\NormalTok{,}\DecValTok{3}\NormalTok{,}\DecValTok{4}\NormalTok{,}\DecValTok{5}\NormalTok{,}\DecValTok{6}\NormalTok{,}\DecValTok{7}\NormalTok{,}\DecValTok{8}\NormalTok{,}\DecValTok{9}\NormalTok{,}\DecValTok{10}\NormalTok{,}\DecValTok{11}\NormalTok{,}\DecValTok{12}\NormalTok{,}\DecValTok{1}\NormalTok{,}\DecValTok{13}\NormalTok{,}\DecValTok{14}\NormalTok{)]}
\end{Highlighting}
\end{Shaded}

Then, we look for a correlation between population and total food waste.
My prediction was that population and total food waste would be strongly
correlated.

\begin{Shaded}
\begin{Highlighting}[]
\CommentTok{\# Look for correlation between population size and total food waste}
\CommentTok{\# Create the scatterplot with a regression line}
\FunctionTok{ggplot}\NormalTok{(foodWasteData, }\FunctionTok{aes}\NormalTok{(}\AttributeTok{x =}\NormalTok{ population\_2021, }\AttributeTok{y =}\NormalTok{ combined\_estimate\_tpy)) }\SpecialCharTok{+}
  \FunctionTok{geom\_point}\NormalTok{() }\SpecialCharTok{+}
  \FunctionTok{geom\_smooth}\NormalTok{(}\AttributeTok{method =} \StringTok{"lm"}\NormalTok{, }\AttributeTok{se =} \ConstantTok{FALSE}\NormalTok{, }\AttributeTok{color =} \StringTok{"blue"}\NormalTok{) }\SpecialCharTok{+}  \CommentTok{\# Add regression line}
  \FunctionTok{labs}\NormalTok{(}
    \AttributeTok{title =} \StringTok{"Population vs. Total Food Waste"}\NormalTok{,}
    \AttributeTok{x =} \StringTok{"Population (2021)"}\NormalTok{,}
    \AttributeTok{y =} \StringTok{"Combined Food Waste Estimate (tonnes/year)"}
\NormalTok{  ) }\SpecialCharTok{+}
  \FunctionTok{theme\_minimal}\NormalTok{()}
\end{Highlighting}
\end{Shaded}

\includegraphics{Exploring-Food-Waste-Data_files/figure-latex/unnamed-chunk-19-1.pdf}

\begin{Shaded}
\begin{Highlighting}[]
\CommentTok{\# Calculate the regression coefficient r\^{}2}
\NormalTok{foodwaste\_regression }\OtherTok{\textless{}{-}} \FunctionTok{lm}\NormalTok{(population\_2021 }\SpecialCharTok{\textasciitilde{}}\NormalTok{ combined\_estimate\_tpy, }\AttributeTok{data =}\NormalTok{ foodWasteData)}
\NormalTok{r\_squared }\OtherTok{\textless{}{-}} \FunctionTok{summary}\NormalTok{(foodwaste\_regression)}\SpecialCharTok{$}\NormalTok{r.squared   }
\FunctionTok{print}\NormalTok{(r\_squared)}
\end{Highlighting}
\end{Shaded}

\begin{verbatim}
## [1] 0.9660028
\end{verbatim}

The regression coefficient r\^{}2 is \textasciitilde0.966, meaning that
96.6\% of variation in a country's total combined food waste can be
explained by its population. This is exactly what I expected to find.

\hypertarget{conclusion}{%
\section{Conclusion 👏}\label{conclusion}}

\hypertarget{recommendations}{%
\subsubsection{Recommendations}\label{recommendations}}

\begin{itemize}
\tightlist
\item
  Countries should improve education efforts around meal planning, the
  consequences of food waste, and proper storage of food.
\item
  Incentives should be given to encourage people to only buy what they
  need or donate what they don't.
\item
  Less-developed countries should implement policies to improve access
  to refrigerators and other storage units that can keep food from going
  bad too quickly.
\item
  Retail and food service businesses should set up procedures to donate
  imperfect or extra food items to those in need instead of condemning
  it to the trash pile immediately. To incentivize this, governments
  could provide tax breaks for businesses that donate significant
  amounts of extra food that would have otherwise gone to waste.
\end{itemize}

\hypertarget{to-check-out-next}{%
\subsubsection{To check out next}\label{to-check-out-next}}

\begin{itemize}
\tightlist
\item
  \href{https://www.kaggle.com/datasets/joebeachcapital/food-waste}{The
  original dataset used in this project}
\item
  \href{https://www.scientificamerican.com/article/how-wasted-food-turns-into-huge-amounts-of-greenhouse-gas/}{How
  Wasted Food Turns into Huge Amounts of Greenhouse Gas - Scientific
  American}
\item
  \href{https://www.worldwildlife.org/stories/fight-climate-change-by-preventing-food-waste\#:~:text=But\%20wasted\%20food\%20isn\%27t,more\%20potent\%20than\%20carbon\%20dioxide.}{Fight
  climate change by preventing food waste \textbar{} Stories \textbar{}
  WWF}
\item
  My other data analysis projects:

  \begin{itemize}
  \tightlist
  \item
    \href{https://github.com/jhould007/Josh-Houlding-Bellabeat-Case-Study}{Josh-Houlding-Bellabeat-Case-Study
    (GitHub)}
  \item
    \href{https://github.com/jhould007/Seattle-Rain-Analysis}{Seattle-Rain-Analysis
    (GitHub)}
  \end{itemize}
\item
  \href{https://github.com/jhould007}{My entire GitHub profile}
\item
  \href{https://www.kaggle.com/joshhoulding}{My Kaggle profile}
\item
  \href{https://www.linkedin.com/in/joshuahoulding/}{My LinkedIn
  profile}
\end{itemize}

\hypertarget{thank-you-for-reading-and-have-a-good-one}{%
\subsection{\texorpdfstring{\textbf{Thank you for reading, and have a
good one!}
😃}{Thank you for reading, and have a good one! 😃}}\label{thank-you-for-reading-and-have-a-good-one}}

\end{document}
